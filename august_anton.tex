\section{REFLECTIONS}\label{reflections}

Memories of the Life of August Anton\\
written as a result of the motivation of his friends\\
published by the writer

(Edgar)

\textbf{To my friends and countrymen!}

Approached from many sides to write down my life's memories as well as the events of the year '48, as far as I was personally touched by them, and to publish these memories, I will herewith fulfill the wish of my friends and only ask for your kind indulgence if my descriptions fail to be elegant. Well then, I will do the best I can.

Once upon a time, many, many years ago, in the old city of Zerbst, in the beautiful land of Anhalt, located in the German homeland, a strong boy was born to an honest draper by his mistress. This boy was later baptized in the Reformed Church and received the name August. Now I cannot honestly remember all the details, only this much was clear to me, that I must be August. Now, the boy grew up, food and drink he loved to taste and when he, because of his young ways, did some dumb pranks, he then often received quite sensible memories from his father or his mother.

With my sixth year, I was sent to a so-called ``Winkel School'' (Corner school) which was run by an old Kantor widow named ``Durre'' and was assisted by her already pretty mature daughter ``Auguste''. Later, when the City Council of Zerbst had a huge stately building erected and designated it to be a public school, all school-aged children, from five to fourteen years old, had to attend it and the ``Winkel School'' ceased operation. At the examination, I received a report card: Accepted into first grade on a trial basis! I went there for the next five years and in the last two years I was the top student.

In the year 1844, I was confirmed and now it was time to choose a profession for life. I would have loved to become a fur tanner; however, I was unable to find a master to teach me. In 1842, I began an apprenticeship at a brother of my mother, who was a master in the fur tanning profession, and I worked at his place during my free time when I did not have school. I had to be confirmed first and before I was able to be confirmed, he died and, since I could not find another master, even though I tried very hard, I began my apprenticeship with Franz Wiegand, the best cabinet maker around. The apprenticeship period lasted four years, 8 June 1848, after having passed my apprentice exam, I was declared carpenter officer and the court assessor in office took my word and handshake and made it my duty to travel to foreign lands, to see other cities and countries, so that, when the time came, I could become a capable master of the trade and a good citizen. Now, that I have honestly done; before I tell about it, though, I would first like to talk a bit about the beautiful land of Anhalt and then about the events of the year 48.

Now, during the beautiful pre-March time, the reigning princes of our German fatherland were the A and O (Alpha and Omega), the sun around which everything turned, and I would like to be permitted to describe some of our has-been landowners. ``Albrecht the Bear'', a mighty warrior before God, was the founder of the dynasty. I suppose there would be little interest if I wanted to recite the entire family tree. We'll talk a little bit about the old Dessauer, the old Mustache, who was a successful general under two emperors and two kings. He did a stupid prank in the eyes of his\ldots Relatives, who were very proud of their ancestors, to where he fell in love with a common girl, the daughter of a pharmacist, her name being Anna Foehsin, and he then married her contrary to all arguments, made her into a reigning princess and lived with her happily for a long time, and also made sure there were plenty of offspring.

To possess her, he did not back down from anything, he even stabbed her alleged lover, whom he found once in a kneeling position in front of her. For this, a common mortal would have been punished as a murderer, but since he stemmed from God's grace, he was allowed to do this. More about his characterization is the following:

As he once, as an Austrian field general, returned from a war victoriously, he was received by the people of a southern city with a fitting festive march (music), which he liked so very much, that this march had to be played at each fitting occasion. He did not want to hear anything else anymore, just the so-called ``Dessauer March''; yes, even in church, when the entire congregation sang holy songs, he would shout out loudly with his terrible bear voice the favorite march, for which the following text was written by a genius:

\begin{quote}
This is how we live!\\
This is how we live!\\
This is how we live each day!\\
With the most beautiful drinking buddies!\\
In the morning with brandy,\\
At noon with beer,\\
In the evening with the virgins in their bedroom!
\end{quote}

Even when he was on his deathbed they had to play this melody for him and he would yell out the text along, until his last breath.

For a successor Franz, called by all his subordinates Father Franz, this was an opportunity, being a quite tight and people-loving prince, who had only the welfare of his subordinates at heart and was against all military actions. His favorite pastime was to travel through the land, on foot, to visit each village and town. He visited each school to convince himself of the capability of the teachers and the progress of the students. He walked each street and lane and talked with all he met, regardless if they were rich or poor, noble or common.

His successor, Duke Leopold Friedrich, was again the opposite. He was people-shy, did not care about the government, left everything to his advisors, who exploited the land and the citizens in the most shameless way. His only passion was to hunt every day, to cross through the woods back and forth was the only hobby he had. No wonder then, that in the year 48 the exploited people welcomed the Revolution.

So still, I want to tell about a princess from Anhalt, who was meant to play a big role in world history. She was selected to be the wife of the Russian Emperor Paul, and when he died, he was a very unimportant human moron, she became Emperor of Russia under the name Katharina II. World history teaches us that she enlarged the Russian State quite a lot and broadened its borders and reached the highest step on the ladder to power. Therefore, she was called ``Katharina the Great''. The famous historian, Johannes von Mueller, says about her: She was great if greatness could\ldots She thought of him without any moral value. He probably means all her love affairs, because when she saw a man whom she liked, he had to be at her will or else it would have cost him his head.

Next to close the characterization of some of the princes of Anhalt, now we will proceed to look at the causes which led to such a general people uprising and in many a bloody revolution. In the year 1847, all food in Germany was very, very expensive. A bushel of rye cost five Thalers, a price which was unattainable, because all wages were so low and all earnings so low, to the point where the mass of the people had to go hungry, while the officials lived high on the hog. In the spring of the same year, I had to go to work, together with several of our journeymen, at a mill for fourteen weeks. The mill was called Amts Mill, located near the city of Zerbst.

There, we did not feel anything of the expensive times; we had plenty to eat, and when we were finished there, we went to Jochheim, a forestry located near the Elbe river, about three hours from Zerbst. The forester's name was Ganzer and his ranger, a unique individual of the first class, was called Fritz Otto. As good as life was at the mill, at the forestry we lived like masters. Here is the menu:

\begin{itemize}
\tightlist
\item
  Early, six o'clock: coffee, sugar, milk, butter, and hard rolls in masses.
\item
  At eight o'clock breakfast: bread, butter, coffee, ham, meat, and beer, and for master and journeymen, brandy too. I, as an apprentice, did not get any brandy.
\item
  At eleven-thirty was lunch: soup, vegetables, meat, roast, salad, bread, beer, and brandy.
\item
  At two o'clock: coffee, sugar, butter, and hard rolls.
\item
  At four o'clock was the afternoon snack: bread, butter, wurst and such, beer, and brandy.
\item
  At six-thirty was dinner: again roast, bread, butter, cheese, meat, beer, and brandy, and at each meal, everything was available in masses.
\end{itemize}

See! such was the life of the officials. At this site, the poor workmen had their mouths and noses wide open. No wonder that unhappiness reigned all through the land and the question arose: Aren't we commoners and workers human beings too, are we only here to carry the load and pay taxes? The old forester Ganzer and his wife were the most deserving people in the world. She, the forester's wife, always asked us to go into the garden and to pick for ourselves berries, fruit, and grapes as much as we liked; there was plenty of all these things there, so we did not have to go hungry. The forester was a very friendly and entertaining man, and each evening, when he was not absent, all people who lived in the house and in the neighborhood gathered outside.

Mr.~Forester sat in the middle, everyone listened with attention. Then he finally began: ``Now I have to tell you a story, you have probably never heard one like it in your entire life, a story of the poisonous tree on the island Celebes,'' and he elaborated long and wide about the terrible consequences the fumes of this poisonous tree had. The birds fell as corpses dead from the air; wild animals, which roamed near the poisonous tree, collapsed dead and the people, who dared to get close to it because of ignorance, also had to die and this way, during the course of many years, so many corpses had gathered near the tree, so that the resulting poisonous smell due to the decomposition of the corpses made it impossible for any living creature to live on the island. And as long as we were there, he told this same story every evening as if it was news, and everybody listened with suspense as if they had never heard it in their life before, and exclamations of amazement and of surprise were loudly heard.The story was told, the congregation broke up slowly and what was the most unusual thing was, that in my opinion in those days, everyone left in pairs. The Forester with his Mrs.~Forester, the hunter, ``Otto Fritze'' by name, with Aunt Gertchen, our master together with the already aging Aunt Minna, the journeyman with the female cook, the servant with the chambermaid and this way also the neighbors, all left paired up. And only I, the apprentice, stayed seated alone with the babysitter; we exchanged our amazements in regard to what made these pairs leave together, one pair going in this direction, one pair in the other direction and all had chosen densely grown over spots. I made a suggestion to her that we both should look for a spot ourselves, but without success, she did not want to and so we just stayed sitting alone together and spent the time in joyful discussions until it was time to go to bed.

I really found life in the country very relaxing, social, we all liked one another, and the Poet is definitely right, when he sings:

\begin{quote}
``Girl from the country, how good you truly are!\\
How the color of virginity suits you so well!''\\
And so forth.
\end{quote}

Now, everything in the world has an end, and so our time also ended and we had to return to the city, had to say goodbye to all the nice people, the beautiful deer roasts and carps, and the masses and fill and excess of everything, and the life filled with hunger began again for the master.

Fall of 1847 had arrived. No one, namely among the common people, had an idea that such important political events prepared themselves, and there even were several indications, which pointed to something very unusual. For example, in our city Zerbst, stood since God knows when a huge sandstone sculpture which was to portray the Knight Roland. Helmet on his head, the sword in his fist. As I was taught, he is supposed to be standing in seven German cities and is supposed to mean, that wherever such a Roland figure is standing, there is a high court, an assize in existence.

The City Council of Zerbst had decided, since the town was very rich, to have a very artistic sentry-box built for Knight Roland and gave the order to execute this plan to a builder named Wilhelm Raedike. He was a very rich and well-to-do man. He employed many craftsmen and sculptors, money was not held tight, only to build a beautiful monument. As the enclosure fell, everything stood there in mighty beauty. Everyone was amazed, jubilant and happy about this very beautiful and expensive building, but yet a murmur went through the crowd, as if something illegal had happened, but no one knew what it was. Then finally it was brought to the surface. A joker, who must have been informed in detail, had written a sarcastic poem and had posted and distributed the same in the darkness of night, without anyone knowing about it. This poem brought forth a storm of surprise, applause, but also upset. It went like this:

\begin{quote}
Many hundred years you have stood there\\
You grey man of stone,\\
You saw as the waves of storm\\
passed you by\\
and also the sunshine.

You stood unsullen,\\
No evil attacked you,

When lovers tasted,
And others threw up,
Your mouth stayed locked.

The wages for such loyalty
Did also not stay away totally.
The Council had decided
To give out a sum of money
To build a sentry-box.

And when the building was completed,
for which much money was wasted,
Then the house was too small.

Oh, my! the master called out,
With him were the smart spirits,
Regardless, he has to enter.

Then the Barbarians hacked,
My heart is still bleeding,
As it did not work out,
Your buttocks into pieces.
Then there was a big hole.

Be patient, you poor
and laden knight,
Time will heal all wounds.
Soon you will recuperate,
Maybe it will grow back again.

All searching for the sinner who so embarrassed an honorable Council and his worthy builder was unsuccessful. The builder had to suffer from all this the most, because, everywhere he showed himself, he was mocked. Even the children in the street called to him: Ham Thief! - Ham Thief! But the best was yet to come, something like this was lingering in the air from the year 48. In the year 48, one of the mockers dared to attack even the reigning Duke, Leopold Friedrich, in a secret lampoon and dared to put his weaknesses and sins on the pillory. Spread all over the entire country the following poem appeared:

The Duke Leopold of Anhalt
Called the Tight One,
Gives money to his subordinates
But not the middle class.

So, not much comes during the day or night
Back from the woods,
And places the well being and luck of the middle man

Into the might of the officials.

The councilmen see themselves in ecstasy,\\
At their strived-for goal,\\
And playing with the possessions of the middlemen\\
A very vile game.

Who already knows the tyrant\\
The chamberlain of Mohs:\\
He is the miller's patron saint,\\
Cause Bung steals wholesale.

Once his millstone is dull,\\
And all the flour is gone,\\
Then he fills the empty trunk\\
With sawdust again.

But of you, Duke, we beg:\\
Put down your hunting clothes,\\
Reign yourself here and hunt and chase\\
The crooks right out of the land.
\end{quote}

The effect which this explicit poem brought forth was tremendous. The feeling of the folk mass for Sr.~Nobility, the reigning dukes, was of the most subordinate type, so that one could not imagine such audacity to be possible. No one dared to talk about it in public, only this much was sure, that one thought of the doer of such things as a very bad subordinate and a very daring human being. No one ever knew who the bad person was. How little was known by the masses of the folk, that such consequences laden events were approaching, I just want to talk about an example.

Beginning March, I was working together with my master at a hat maker's - the folks called him head shoemaker, a rich, respected citizen, Fritz Rost was his name, his nickname was Always Green, because despite his white hair, he still executed some great pranks, just like a youngster. Now, as noon approached, we - my master and I - the apprentice, were invited to eat lunch. There were other guests at the table. Mill owners from the surrounding areas, who had come to town on business. All were very rich, well-to-do people, of whom you would think should be well informed about daily questions and the events in general.

Soon one talked about politics and the February Revolution in France, where they wanted to create a republic and chase away King Louis Philippe. These were the main subjects of the discussion. All agreed that the French were splendid fellows and that this type of thing would never be possible here in Germany, especially in Anhalt. Because, they said, if someone here would speak up in public and speak with uncalled for words, soon this person would be grabbed by the collar and stuck into the hole and the others would rub their hands shut. Here one young man screamed out: it serves him right, why can't he keep his mouth shut. Rich, respected citizens spoke this way, they had been all around the country, of whom you would think, that they would be well informed about all public affairs.No one had any premonition that a total overthrow of all existing political situations would be so near.

It was one Saturday, 17 March. As the barber, Franz Habusch was his name, came in the morning into our place of work to shave all the apprentices, as he usually does. When he was finished, he stood up and yelled:

\begin{quote}
Now, all you apprentices, and also you journeymen, form up in ranks, I have to tell you all something very important. - Now this evening, everything will break loose!
\end{quote}

``What?'' we all yelled.

\begin{quote}
You will see, so be ready. You journeymen will have to fight for freedom and right, you will have to sacrifice everything, even if it has to be your life. Now the time has come to topple the tyranny and to build a free people's state. You apprentices also have to come, you have to break all the windows and you have to stand by readily to help. Once again, I need your word and handshake on this, that you will do your duty. I don't have time anymore, I still have much to do today and have to prepare as many people as possible for the big events. Again, good morning and you all come!
\end{quote}

The journeymen and us apprentices stood there and did not know what to think of all this. The day went by with hope and hesitation. We ate lunch, nothing suspicious happened on the streets and we all thought: In the end, he probably just joked around with us.

As we sat together with our master at the supper table, which is eaten an hour earlier on Saturday, at six o'clock, all of a sudden it all broke loose in the streets. A very shrill whistle, which went through your bones, running, shuffling, drumbeats, so loud that the food got stuck in our mouths and we all looked speechless at each other with horror and surprise, and we got up and wanted to stumble out into the streets.

Then our master, who as master cabinet maker was a man very loyal to the government, got up and commanded us to stop!

\begin{quote}
Journeymen, he said, there is something illegal in the coming, I warn you, don't participate, but stay quietly at home. Because something like this as is in the coming this evening, has only bad consequences and will often act as a holddown in the entire future life of a person. Therefore, please hear my warning, I speak to you as an experienced man and as a friend. And you, you apprentices, I forbid you to leave the house, you stay at home and go to your bed.
\end{quote}

It was easy to preach for him, the noise in the streets became louder, so that the journeymen, who stood there shaking because of excitement, didn't have anything more urgent to do but to put on their skirt and their hat, and stormed out into the chaos. We, the boys, looked at each other speechlessly, the calls in the streets became more and more enticing and so we then also ran outside! may it cost what it will. We sneaked around between the houses, through the so-called corridor and climbed over the corridor gate and, outside we were, on the street, right in the middle of the chaos.

What we especially noticed right away was that everybody was smoking, since until now a strict police ban punished smoking on the street with a heavy fine. Everyone,right down to the youngest boy, who could get a hold of a smoke stick and if one could not afford to buy one, one was given to him, everyone in their own way, everyone marched loudly and screaming down to the market square.

A huge congregation. A printer, once already punished for political activities, his name was Smoker, later became the editor of a German newspaper in New York, called the ``Evening Post'', a big, majestic figure, was president, i.e.~the spiritual leader of the whole thing; while a tailor named Esterheld was the actual commander of the whole thing. As things were really rolling, and the president encouraged everyone to stick together, to arm themselves and to fight for freedom and equality, to win and, if necessary, to die, an unexpected incident happened. Police Chief Brumby, who since a long time already had led the police regiment with an iron fist, found himself called upon to suffocate the revolution while still in the bud. He gathered all normal thinking citizens around himself, and as wide as the street was, marched in closed ranks towards the revolutionaries. As soon as this was known to the president, he immediately ordered the commander, tailor Esterheld, to fight the group to the death and to destroy them.

Advance! he ordered. On them! On Them! Down with the dogs! And now it started, we youngsters threw rocks; the bigger ones knocked off all the hats and if people did not run away in a hurry, they were crushed under our feet, and the one who could hold his skirt together and run the best was police chief Brumby. Now they sang again and later they thought about things further. Everyone screamed as with one voice: ``We want freedom of the press'', and all during the following days, everyone yelled, we want freedom of the press! After several speakers had spoken, it was suggested to hold a parade through the town, so that all the citizens could be encouraged to participate. Said, done. It began, singing songs about freedom, everyone marched through the whole town, under the lead of the already mentioned tailor Esterheld. We youngsters threw rocks into the windows of the opposing people's homes. Later, again a meeting at the market square, still various speakers held speeches until the president of the event ended it and asked all present to go home quietly for now, and to come back again tomorrow evening, until then he would prepare everything, as was in his power, he would promise, though, that tomorrow evening many important things would happen. Well, another cheer! Another call for freedom of the press! and everyone went home; also us, the apprentices. Not to get into trouble with our master, we climbed again over the corridor gate and laid down in our beds quiet as a mouse. Everything was over for the evening of the 17th March.

Now Sunday came, the 18th of March. Everyone waited for evening, of course our master tried to influence us not to participate in anything, my mother and father also warned me not to go there, but all was for nothing. When evening came, everyone was back out in the streets which were all filled with people and off it went to the market square. It did not take very long when the president and commander appeared and ordered to march forward, after everyone was lined up in rows. We youngsters marched ahead armed with rocks. From the market square we went across the old bridge, then at the point where Akensche Street begins, everyone stopped. This is where the highest spiritual person of the city lived, Muennich. Stop! sounded theIn one wing of the house, they chopped all the furniture into pieces.

``Get back in row! Hurry! We still have lots to do,'' he said as we marched again, drums ahead, singing songs of Freedom, out through the gate of Akenschen. Along Casper Street, to the house of the royal Forester, Sitzenstock, a most disliked man.

``Stop!'' was called out here, ``Back to work, guys, advance!'' A hail of rocks was the answer. All the window panes were smashed.

``Go on, march forward, men, get to work!'' yelled the commander. And now it began. Across the street was 2 Ladder House, where you got ladders and fire hooks, and six to eight men grabbed the fire hooks and rammed into the house door. The driving mass of people pushed their way into the house, smashing everything in sight, piano, beds, and destroying everything, until the order was heard:

``Enough! Line up! Let's go to the Duke's residence, let's go to the head official Schultz! March forward!''

Rocks smashed all the windows.

``Order, go to work and a rain of rocks,'' smashing everything.

The men pushed their way into the house, smashing everything into small pieces, until the order was heard.

``We have a lot of work to do this evening. And forward.''

As soon as the order was given and the mass of people had started to move, a big and well-built man threw himself against the procession and ordered with a commanding voice, ``Stop.'' Everyone stood like a building and stared at the person giving this order, the person who would dare to stop the march from continuing.

``Stop, you citizens!'' he again called out. ``I myself am the high official Schultz, the one you were trying to see to destroy his possessions, which he earned for himself over the years through lots of hard work. Think about what you are trying to do. I admit, there are many bad situations at present, like the dividing up of public property, but I am only the lessee and just took care of my family, and if you destroy everything, I have worked for nothing my entire life and you have not gained anything by it. I wanted to make a suggestion to you. You citizens!'' he yelled. ``Select your best man, make a circle, and let us two fight it out together. See, this way we can solve this situation in a very honorable manner.''

Total silence followed his speech. The leaders stood together and after a short consultation, they said, ``Let him go. He spoke like a man.''

``Okay, left about!'' The march moved forward again, following the orders, right through the garden of the castle. Singing, drum beating, and with much noise in front of the main guard. The entire guard stood there pointing their guns, but neither hand nor foot moved, no one was there to give an order. The court advisor by the name of Pannier, was the only man in town, who in very unusual situations could give this order in place of the Duke. It was said that he was happy to see that the city officials, especially the head mayor, are having quite some trouble on their hands.

``Okay, the march began to move forward again, through the castle grounds, across the old bridge, always with the call: Boys, watch out! Don't miss any windows, and we really did our duty.'' At the market square, in front of the house of a Jewish bank president by the name of Wolf, we stopped.

``Attack!'' And everything that could be destroyed in a short amount of time was destroyed. No window remained intact in any house, everything was smashed. All the furniture was smashed into pieces.

``Advance!'' the new order sounded out. ``Line up! Let's go on to the Ham Thief!''

``Okay, off to the Master builder's house, who had played such a prank on poor Roland.'' When weEverything was more to do this and sounded the command. Hurry! Then, get to work, but hurry! As we arrived, we were told: Stop! The command was given to march down the broad street, and when we reached the house, halt. As I was told later, it was supposed the mayor could not stay in office any longer.

So, the first priority for a group of men was to climb on top of the roof and take it off. With the help of the ladders, they climbed to the roof and soon the roof was gone. Now smash windows, break the front door, drag everything into the street, set fire to the furniture, and then destroy everything in the entire house. Then we proceeded to the barn for the horses. All horses, and there was quite a number of them, were taken out and then chased around the street like wild animals.

After the madness had quieted down some, off they went under the ground in the wine cellar and that's where the whole army found their destiny. All stored stock, and there was so much of it, ten times as much as to get them all drunk, was brought upstairs, and what happened then, dear reader, I probably don't have to describe. It did not take long until the freedom fighters lay around the floor and the fun had a natural end for Sunday evening, 18 March. Whoever could walk went home, and whoever could not walk, slept off their drunkenness. These were all the brutal and illegal things which happened here.

The next day, 19 March, it was a bit different. The good thing about all this was that in those days in March 1848, in each village, in each town, at the same time, all hell broke loose, it was just like a thunderstorm was brewing, it was in the air. The farmers on the farms beat up their mayors and proceeded armed with hayforks and axes towards the residence, in front of the Duke's castle and screamed for freedom and freedom of the press.

Again we heard the order: Now then, the 19th of March, a Monday was approaching. No one felt like going to work, but in our workplace at the Duke's master cabinet maker, everything had to run its normal course. We had eaten lunch and the journeymen and us apprentices took a short break in front of the house door. All of a sudden, this unusual terrible noise. A carriage from the Duke's house, with four horses and four riders in front, came racing through the city gate; then onto the market square in a terrible hurry and in front of the house of merchant Richter, where the head councilman for appeals, Ludwig Habicht, lived upstairs, they stopped, and after they found out that the councilman was not at home but was at a pub across the street sitting there amidst a circle of happy drinking buddies, they then went there.

The Duke's official, who sat in the carriage, presented an order to Habicht, which read like this: In the name of the reigning Duke, we are to bring the head councilman for appeals, Ludwig Habicht, immediately to the residence of his highness. Dead or alive, read the order. There was no resistance. He climbed into the carriage, the trumpets sounded out and in a flying hurry they went to Dessau. Arriving there, he was immediately brought in front of his highness, who addressed him with the following words: Habicht, I have heard that you are the only man in the whole land, who can calm this storm and reinstate law and order. Answer, can you do this?

\begin{quote}
\textbf{Habicht:} Yes, your highness, I can do it.

\textbf{Duke:} How do you want to begin? How do you want to do this?\\
\textbf{Habicht:} Highness! That is my business.

\textbf{Duke:} What do you ask for?\\
\textbf{Habicht:} Unconditional power over the military and civilians. Over life and death. The Duke's house is an exception.

\textbf{Duke:} There is no other way?

\textbf{Habicht:}No, your Highness.

\textbf{Duke:} Well then, you should have unconditional power; and with shaking hands, he wrote out the demanded power of attorney and signed his name to it.
\end{quote}

Now, Ludwig Habicht was close to the Duke. The most powerful person in the land, who can calm this storm and reinstate law and order. And all had to follow his orders. All the ceremonial things he didn't watch. He just wrote: I am ordering the General, the Major, or Council such and such, and everyone had to obey. The first thing he did was, that whenever he made an appearance in some village all through the land, he ordered the entire population onto a large open area and addressed them as follows:

Citizens! I have come here to give you advice as to the best way to go about obtaining law and order in the fastest way possible. If you want to stand by me, I promise to buy you a house in which you can live very well, but I cannot do it alone. You have to help me. Do you want to? So answer with yes? And all yelled out as in a choir: Yes! this we want with heart and soul! Thank you, he said, for your promise, so let's get to work. You have to get yourselves organized. Most of all, you have to build up a citizens' army. Each part of the city needs one company. Select a major for each company and overall a city commander. I will supply the weapons. Well then, let's get right to work! Most importantly, create law and order. I will see you again soon. For now, farewell. I don't have any more time now, I have to get around the entire country and personally speak with all the citizens. So, again thank you for your willingness. Farewell! And off he galloped inside the carriage, accompanied by the sounding cheers of the crowd.

Now, immediately they got organized. For each part of the town, they appointed a citizens' army major, mostly fat, heavy citizens, and for all of the army a city commander. A pharmacist named Busse was elected, and after the necessary guard posts were chosen, the main people, their companions, were designated as lieutenants, majors, and officers. Then they went to the Duke's castle where they received the weapons, and since each citizens' army soldier was very proud of his newfound honor, everything went very well and got on its way. The entire town and surrounding areas were patrolled, and I would have advised anyone not to go against the citizens' army. With all of this, the minister Habicht had hit the bull's eye, and just like with a major stick, the revolution here had ended, and law and order became again the daily routine. Liberals, freewilled delegates were voted in by the people's representatives, and it seemed that the greatest harmony existed between the Dukeand the common people. Now they exercised, marched, patrolled, never-ending meetings and festivals were held and at all the festivals and gatherings there was never a shortage of beer, beer, and again beer, and music and dance forever, and everything was free, without a price to pay by anyone, it was for everyone, all the class differences went out the window. Everyone was on a first-name basis, and if someone said to a minister or other high official: Sir! that person would stand there all astonished and say, Please don't do this, we humans are all brothers. If there was a gala ball, and there were many, and people of all classes attended, princes and princesses, the highest noble ladies and gentlemen were present with their entire families and sat in a row with the common citizen women and girls, the maids and the farm girls. Everything appeared in the greatest harmony, and if they played music for dancing, the journeyman danced with the princess, the common soldier with the citizen, the prince with the maid or the farm girl.

See, this is the way it was in '48, life like in paradise, only too bad that it did not last long. The Duke was willing to fulfill all the wishes of his subordinates. Here is another example. The Duke's next oldest brother, Prince George, lived in Florence, Italy, and since he had a yearly allowance of 18,000 Thalers, the citizens of Zerbst would say: We have here the most beautiful castle in the country, Prince George must move here in our area and spend his money here, and immediately the Prince and his family with eight children moved here. Since I always had to work as an apprentice at the master cabinet maker's at the castle and this way we also had to refurbish his furniture, which was scratched up from the trip, I had plenty of opportunity to meet the Prince's entire family. The Prince himself was a likable, very friendly old man, who walked around the whole town, ate lunch with the commoners, etc. The oldest daughter had the title of Princess, the other children were only counts and countesses, because the Prince in his second marriage had married a commoner, a daughter of a musician from Dresden, who he promoted to Countess v. Erdmansdorf through the Emperor of Austria, in order to make the marriage possible. All were very friendly towards us workers and the greatest fun we had with the servants, who were all Italians.

One Sunday, our precious Mother of the Land, the Duchess, had announced her visit to Zerbst and at two o'clock in the afternoon, the entire population of the town and the surrounding villages marched to greet her on the country road. The civil army with musical notes and the farmers all on horseback, they all awaited her at the ``Kruemling'' (bend), named this way because the road makes a bend there. We saw her carriage already from far away. As she had approached far enough and a cheer of thousand voices came out of each one's throat, she stood up in the carriage and said: I greet you all from my heart and invite you to come along with me to the rifle range, where we will all have some fun. Another cheer and away we went, with melodies and song, the precious Mother of the Land in our midst, off to the rifle range. Arriving there we found 39 huge barrels of beer, not the usual ``kegs'', but big huge barrels. Every one had been tapped, crowned and plenty of glasses were available. First of all she was handed a full glass and then everyone present stood around the carriage. She stood up, toasted with as many as possible and then, swinging her glass, she shouted in a very loud voice: This! drink to your health, myChildren of this land! A cheer from a thousand voices sounded out, and cheer, and cheer again! Everyone was quite taken in by this big, pretty, and proud lady. She gave a sign and asked everyone to be quiet since she wanted to speak, and now she spoke with a loud, far-reaching voice:

\begin{quote}
Precious children of this land! I invite you for this evening. A gala dance is to be held in each inn. I will take care so we have good music and refreshments. The city commander will tell you the details. And now, live well my children! This evening we will see each other again. Come, come all!
\end{quote}

And off she rode, accompanied by the sounding cheers of the crowd.

The town commander stood on the balcony of the rifle house and again invited all and he also gave the names of all the inns (restaurants) where the gala dances would be held. After all the barrels were empty to the last drop, while dancing and laughing and singing and jumping around, the crowd broke up to get ready to gather again that evening in the described inns. Then they really got going and as I already indicated earlier, all social classes were represented and in each inn, the royal lady made an appearance. She sat down at a table in the middle of the dance hall and watched the dancing crowd around her. And when the right time came, she got up, took a glass of beer, toasted with all people standing close to her, and yelled with a loud voice:

\begin{quote}
To your health, my children.
\end{quote}

And then she removed herself, accompanied by the sounding joyous noise, to visit another dance hall and to make her appearance there. There were plenty of refreshments, and when the joyous crowd left in the early morning hours, everyone had just one wish:

\begin{quote}
Oh, if it would just always be like this.
\end{quote}

See, this is how the revolution was killed in our beautiful land of Anhalt. No one ever thought of rebelling against the authorities, and everything again became normal, while in other towns, i.e.~in Berlin, hundreds of dead remained on the barricades and people called to Friedrich Wilhelm IV, also called the old Lehman, as he spoke to his citizens in the courtyard of the castle, with their heads covered:

\begin{quote}
``Fritz the Slip-up''
\end{quote}

So, everything went along its normal way in our town. At each festivity, there was beer and beer again, this was ordered by the citizens themselves. They elected representatives who made liberal laws and one year later, his highness, the Duke, said to his almighty Minister:

\begin{quote}
Habicht, I don't need you anymore, I believe I can get along without you from now on.
\end{quote}

And with this, he was relieved of his office, he kept his minister's wages though, which were quite high, and he was gone and forgotten.

Now then, to speak again about myself, I was knighted on the 8th of June. Now I was a journeyman; the beatings and kicks, which were a big part of disciplining an apprentice, had come to an end. I wanted to get going and travel out into the world, when a cousin of mine, who lived in Lynnsburg, arrived here. His name was Redlicn, a man of the world of first class. He had purchased a restaurant in a Prussian village named Scheinitz near Loburg in the province of Brandenburg. He said:

\begin{quote}
August - you are going with me, you'll have it made there.
\end{quote}

Well, as you know, youth is very receptive to ideas, so I went with him. I lived there about half a year. I saw again how beautiful life is in the country, so peaceful and comfortable. I saw how pleasant and friendly the girls are in the country. But all of a sudden, it was all over. The thought came to mind that I cannot rot here in a village in the country, I have to go into the foreign world, travel into the wide world and seek, as promised, to become a workman one day. I went back to Zerbst and on March 6, 1849, I took off to\ldots Dresden, with my knapsack on my back, my walking stick in my hand, together with a travel companion, stemming from Halle, a travel book about Dresden in my pocket. My only sister Johanna, who died already in 1856, the wife of a rifleman, accompanied us for about an hour. My dear mother, deceased, had given me the large sum of five Thalers for traveling money. Now then, everything was taken care of. Goodbye! Goodbye! Now a kiss, come back home healthy and off we went, filled with happy courage, off into the wide world.

We took off via Aken, Gotha and Leipzig. Here we parted, I went to Eilenburg, a very lively industrial town in Nordsachsen, just at the border to Sachsen, and worked there. There I worked, fit and in high spirit, until all of a sudden, as we went out on Sunday and ended up in the highly praised Sachsenland, we saw everywhere, much to our amazement, the following writings posted:

\begin{quote}
``Off to Dresden. All men able to use a gun, off to fight for freedom and rights.''
\end{quote}

On Monday morning, my coworker Fritz Zachnit from Schlesien, and I quit our job and off we went to Dresden. Arriving in Dresden, we stood at the bridge over the Elbe River, heard the gunfire and noise of the grenades in town, and we were undecided about what we should do. Then, all of a sudden, there was a division of Turners from Sachsen, waving a flag and all carrying guns.

``Onward!'' they called out to us four workmen. Two weavers from Chemnitz had joined up with us, and we marched with them. As soon as we arrived at the bridge, there was a Wenden army marching towards us, and as we wanted to turn back, we saw with shock that one could not pass over the bridge there anymore either, because the Prussian regiment Alexander came marching in. There was no more escaping now. Of course, the Turners tried to defend themselves, but all was for nothing, they were trapped in a terribly narrow place. The Wenden proceeded with pointed bayonets, one could not even think of successfully defending oneself. Most of the Turners, as they hung on the bridge railing, had their hands hacked off and they fell backwards into the Elbe River. The rest were taken prisoners and thus also us four workmen. Since we did not have any weapons and our travel papers were in order, they brought us to the police headquarters. There, they asked us where we were going and they transported us out of the town.

What we saw along the way was horrible. They led us far around the barricades, where there were still many, many armed people. Across from the Elbe bridge was a big hotel, ``To the City of Rome,'' there - everything was destroyed, on the outer wall, there wasn't room enough to put one hand width between all the cannon and grenade hole marks. The hanging signs in the street were perforated with bullets. The fight had taken its biggest toll at the gallery. And terrible stories were told about the Wendens, that they entered the houses through the sewers and murdered the entire population, defenseless women and children.

So then, us workmen, shaking from excitement, were taken across the border and we were told, never to come back if we treasured our life. We marched forward to Pirna, Schandau, into Saxon Switzerland. We could not climb the Königstein (mountain) since the exiled King stayed there. The Saxon Switzerland is such that if someone who has never seen this part of the country could hardly imagine it. The rolling hills, the shadowed valleys, a paradise on earth. This prompted me to take onwork in Sebnitz, right at the Boehmen border. From there, we could take excursions each Sunday to Saxon Switzerland. I had gotten something anyway from Dresden; I had bad veins on my right leg and in the shuffle on the Elbe bridge, I don't know how, maybe from a cannon wheel or a push against the bridge railing, I was injured, and this injury, even after healing, broke open each year for the rest of my life, and each time rendered me unable to work for four to six weeks, and just now, the wound has opened up again.

``What made my stay in Sebnitz even nicer was that I met a relative from Zerbst there, by the maiden name Mathesius, who was married there to a weaver named Wenzel. Now then, as everything has to end some day, my time came too!

I once again tied up my knapsack and walked to Prague, where a brother of my mother lived already for thirty years, and I had promised my mother I would visit him. Well then, goodbye, all you dear Saxons, now it's off to Bohemia. And what horror tales they have told me about Bohemia. But nothing was true. My first stay overnight was in Schluckenau. I have never met better and nicer people. Then I went to Rumburg, Tabor Bohemian Leipen. When I arrived there, something very unusual happened to me. As soon as I had entered the Inn, the innkeeper came towards me, embraced me, hugged and kissed me and said, welcome, my son! I have waited for you so long. You are my son! Father! she called out to her husband, see here, now we have gotten a son. I stood there speechless, I did not know what to say. She, however, pulled me down on a chair beside her and said, You stay here with me, you'll be my son, I'll get you work and evenings and Sundays you'll be with us, and if you don't want to work at your job anymore, you'll help us, we have no children and when we die someday, it will all belong to you. I was speechless over all this kindness. I got plenty of food and drink and at night I had a nice bed. The next morning, when the same love began all over again, I got myself together and said, Mother, when I was still in school my mother always told me: Boy, when you go far away someday, then you'll first visit my brother in Prague. See, mother, I said to her, and I promised her that and I have to keep that promise, I have to go to Prague and then I'll come back to you. Yes, she said, I know all that already, once you are in Prague, you'll never come back to us, since you have already promised your mother, then go on, but you'll not take your knapsack, I'll send it to you there. Here, she said, is the address where you can pick it up.

Said, done. I left for Prague. At Prague I went to the carpenters' Inn, located at the Horsefair. A huge, mighty Inn, where hundreds of men and women sat and drank beer. As soon as I sat down, a waitress came towards me with a mug of beer, she embraced me, hugged and kissed me and said: You are my bridegroom. I was totally astonished, and thought: Is everyone crazy here in Bohemia? I drank my beer and thought about where I could find my cousin. When I saw that an old man sat next to me, I asked him if he knew the company Boulange \& Burdan. He replied that he knows them, and when I asked him if he knew the foreman there, who is my cousin, he said yes and he would take me there. Said, done. We got up, he took me there, a short piece before we got to the Moldau bridge. Here, he said, go straight through, then cross the courtyard, go into the workshop and there you'll find yourcousin. I went to the workshop, looked left and right, then I finally saw the face; I walked towards him and said:

Good day, cousin!

Cousin! he said, where do you come from?

From Zerbst, I said.

And your mother, what's her name?

I said, my mother's name is Sophie.

So, he said, then you're welcome, cousin. He took off his apron and led me out to the courtyard up a stairway to his family. There was such happiness, the children jumped all around me and were so happy. Then the talking about the former home began and when nightfall came, he took me back to the Carpenters' inn, since he had no room for me to keep me overnight. So, I stayed one whole year in the old town of Prague. Ups and downs and everything that I had to experience there, it would fill a book all by itself. Just this much, I found a master, who really taught me the art of cabinet making. All the Bohemian, Moravian, and Polish noblemen came to him and wanted him to make their furniture. There was no arguing about price, just make the pieces as good and as nice as possible, the price did not matter.

I met a good friend, Julius Poesch from Graz, and with him it started all over again.

``Goodbye, you beautiful city of Prague! - which has a bridge made of stone, - the Hradčany up on the hill!''

We took off for Linz, located by the river Danube. After we worked there three weeks, we went to Innsbruck, the capital of Tyrol. Several things happened on the trip there, and they are worth telling. As we walked in Salzburg across the bridge over the Salzach, a beautiful wagon was coming towards us. Riders rode ahead of it, then the beautiful wagon, everything was covered with flowers. In the back were several figures, then riders followed, everyone in some foreign folk dress, so that we stood there speechless and thought, that it could probably be the Sultan of Turkey. Downtown we found out, however, that it was King Otto of Greece. After we had eaten lunch at a convent, we wanted to leave for Hellbrunn, we ran again into King Otto. He went, the Queen, the wife of the dethroned Franz, the pottery maker, as his subordinates called him, accompanying, she was carried to a resting place by the grave of Mozart, who was buried in Salzburg. After we had seen everything, we went ahead to Hellbrunn. The Duke's bishops of Salzburg have created a paradise here, the likeness of which you'll never find anywhere on earth. In all kinds of temples, out of the mouths of thousand monsters, water was spraying and because the wealthy visitors, who left plenty of tips, should have some fun, people would tell this and that wearing the most solemn facial expressions. The favorite people got a wink, and from a thousand different sources, out of the nose and mouth of monsters, from the air, out of the ground, the water sprayed in thousand streams down onto the poor devil, they cried out of anger and the rich visitors laughed themselves half to death. Away we went, turned our back on Hellbrunn, and stayed overnight at a farmhouse and the next morning we went to Oberbayern, Reichenhall, Berchtesgaden, across the steep mountain ridge called Hirschbühel.

Since it takes nearly a whole day to get up there, again some friendly people took us in overnight and the next morning we started out. Our hosts had given us plenty of food and high into the clouds we climbed. We proceeded to Zell am See, that's where the area Pinzgau begins, whose inhabitants, as it says in the song, love to hike. The entire valley all the way to Mittersill has no towns, only a few farm stores.Each local, however, is mad if you pass his house by. Here is a paradise for pig hunters; we, however, since we wanted to keep going, marched right on. The priests gave us money, and the rest of the population gave us plenty to eat; this was a beautiful life. At the end of the valley, we had to again go across a mountain pass, the Platten, and then to Tyrol into the lovely Zillerthal.

Everywhere people with vacancies. Men, women, and girls called out to us: ``You boys, come on in, stay with us,'' but we could not go with all of them. We wanted to keep going and so we went through the upper and lower Inthal to Innsbruck, the capital of Tyrol, where we started to work.

To characterize the behavior of the police here in free America and there in Tyrol, here is an example. One day, a young man stemming from Bavaria came into our workshop and began to work there. And since he had not learned very much yet but was willing and friendly and sociable, we helped him through it as best as we could. One nice evening he said:

\begin{quote}
You are so good and friendly towards me, I would like to reciprocate, come with me to the Inn. Look here, I have a big sack full of Bavarian Gulden (money), let's have a great evening.
\end{quote}

Okay, let's go.

\begin{quote}
Waitress, give everyone a glass of wine.
\end{quote}

Now, this one was downed, and then another and another, and as we were drinking right along quite well, the master came in. To see us sitting there and tell the waitress:

\begin{quote}
Bring each one of the gentlemen a glass of wine,
\end{quote}

took only as long as the blink of an eye. And so we all kept drinking until all the carpenter journeymen, except myself, had fallen under the table.

\begin{quote}
Line up!
\end{quote}

I yelled, helped all of them to their feet.

\begin{quote}
Forward, home!
\end{quote}

All stumbled forwards, arriving at the stairs, they fell down the same, I myself slid down carefully. After I got down I saw that each one of them was held by the arm by a policeman; one wanted to grab me too. I said:

\begin{quote}
Not necessary, I am sober.
\end{quote}

\begin{quote}
Where do you all live?
\end{quote}

they asked. I answered:

\begin{quote}
Riesengasse No.~9.
\end{quote}

\begin{quote}
Come,
\end{quote}

I said,

\begin{quote}
I'll lead the way.
\end{quote}

When I arrived there, I unlocked the house door, turned the light on in our workshop, and the people upstairs also brought lights down. The policemen pushed all the sawdust together into one big pile using the butts of their guns, laid down carefully all drunks in a circle on top of it, and after they were all bedded down well, they left. Now everyone would like to make a verse about what the police here would have done in this case.

The following morning we were all fired. The master said:

\begin{quote}
Journeymen who cannot even say an ``Our Father'' (prayer) and get drunk, I cannot use here.
\end{quote}

Again, off we went, my friend Julius and I took off for Triest. Off we went towards Brixen, across the Brenner (mountain pass), a trip which I will never forget in my entire life. Through ice and snow up to your ankles we stumped through tough mud, and at night we slept in wet clothes in some barn. The clothes were in shreds and just hung on our bodies. So then we arrived in Brixen. We went to the police and wanted to go to Venice, but the official talked us out of it. He said:

\begin{quote}
Radetzky is occupying Venice at the present time and the hatred of the Italians against all Germans is so strong there now that they would kill you. Go through the Pusterthal (valley) to Triest.
\end{quote}

Off we went through the Pusterthal, then across a very steep alpine pass to Ampesso. From there to Treviso, Belluno, Conegliano to Triest at the Adriatic.Sea. The area is a paradise, only it's not for poor young workmen. If it had not been for the Jews and the Catholic priests we would have starved. We often ran into soldiers who were returning from the Hungarian Campaign. I have never seen a picture of greater sorrow and suffering. All were barefoot, the officers wore women's skirts, their swords dangled from a string, that's how they walked quietly along. The Austrian government was embarrassed to let these soldiers walk into the military post like this, therefore, they housed them in villages near the Adriatic Sea until they had new uniforms and had fattened up a bit.

So we finally arrived at Trieste on Saturday after having experienced some hair-raising things. Starving, ragged, looking like bums. The innkeeper where we turned in, the inn was called ``To the German Knight'', showed no mercy. He said, so much trash comes here, look for work and then I'll keep you. Okay, let's look for work. He sent us to a master named Koselik, named after the old railroad crossing barricades. There, hundreds of workers were employed, mainly in shipbuilding. As we arrived there, the foreman, a German, said to us:

\begin{quote}
We need many workers, but the German workers have all been let go because they did not want to work till eight o'clock anymore. I cannot employ any Germans until the owner gives permission and he won't be back till Monday.
\end{quote}

So, that was it for us for now. Hungry and freezing we returned to town, tried to beg for a piece of bread, but we tried without success. We told the ``German Knight'' about our mishap. He said:

\begin{quote}
Have you already been to the Sachsen?
\end{quote}

We said no. Then he described to us one more time where they lived, on Laibacher Main Street. It was already getting dark but our needs pushed us to go on. Then, finally we found them, a big, huge, grandiose house of the firm: ``Laboratorio di Faleynamie e Billardi e Eschebach e Ehrenberg.'' We rushed upstairs to the office. We said, two out-of-town carpenters are asking for work.

\begin{quote}
We have work, answered the man in the room. Where do you come from?
\end{quote}

He asked. My travel companion answered: from Graz, Steiermark; and you? he asked me; I said I am a Prussian.

\begin{quote}
Where in Prussia, he kept asking.
\end{quote}

I answered: I'm really not a Prussian.

\begin{quote}
Where are you from then, he continued to ask me.
\end{quote}

From Anhalt, I said.

\begin{quote}
So, from Anhalt, he said, where in Anhalt, he asked me.
\end{quote}

I said, from Dessau.

\begin{quote}
From Dessau! he screamed surprised, that makes you a fellow countryman of my companion.
\end{quote}

He tore the door open and yelled:

\begin{quote}
Ehrenberg! Ehrenberg! come in here, and he continued, I am introducing you to a fellow countryman here.
\end{quote}

General surprise, then happy handshakes and the agreement, that we would have work on Monday. Well, this is all very nice and good, but the German Knight does not want to keep us without money. They said, go to the police, leave your travel books there and bring us a Charta di Permanenza, then we will give each of you an advance of one Gulden. We went and got it quickly, and we each received one Gulden as an advance and, feeling rich, we marched over to the German Knight. He must have already seen it in our faces and asked:

\begin{quote}
What will it be, gentlemen?
\end{quote}

A sour beef roast and a glass of wine! I believe that this would not taste any better to a king, as it tasted to us; afterwards, we got a nice bed and slept like the gods, after such a long period doing without food and bed, wonderful.

Now we felt great there till February 1851. My travel book expired on March first, so I had to keep traveling; I went to Laibach and from there to Graz, staying at my friend Julius' parents, where I spent a few happy weeks among nice people and experienced a real beautiful Mardi Gras, as is custom in the South. Then I went to Vienna.located at the beautiful blue Danube river. I was there another four weeks and worked at the shop of a harmonica maker. From there again to Prag, stayed at his cousins, visited old acquaintances again, and then I went via Dresden, Leipzig, Halle to Zerbst. Late in the evening I arrived at father's, mother's and sister's and as is expected, I was welcomed with open arms. And the next morning, early at 5 o' clock, I already had to leave for my draft appointment in Dessau. My lottery number was No.~13. During my physical examination I was declared unfit for the military. ``Anton is exempt due to his bad veins,'' the verdict read. So I was free again, was able to travel again into the faraway world.

I first worked a few months in Dessau, where I had met a regional representative to whom the government had given the choice of either being investigated and punished or to emigrate, and since he had already been in Texas in New Braunfels during his youth, he decided to move there together with his family, wife, son, and daughter. Since he was wealthy, he wanted to found a colony there, and for this, he wanted to take with him an architect named Freisleben from Dessau, a toolsmith Herrmann from Chemnitz/Sachsen, a gardener Simon Schmiedt from Dessau and little old me. We traveled through Magdeburg, Braunschweig, Hanover, and Bremen. In Bremen, I met, by accident, a fellow countryman named Leopold von Hammer, a most kind person. He had been an Austrian Hussar Lieutenant and had to quit his job because of too many debts. He lived two years in Copenhagen and was now already six months in Bremen. The Bishop from Münster stopped him. He told me: ``Fellow countryman, don't go to Texas. I want to make a suggestion to you, I have a cousin in Mexico, a very wealthy man, he had been a longtime government official in Mexico and obtained 6000 acres of land from the government and would like to start a German settlement on it. We will land for nothing, money on top of it at the beginning and you will go there with me.'' Said, done; influencing helped, I went with him to Texas. Though, the plan never developed into anything, and the gentleman borrowed money from us and cheated us out of our clothes and one nice day he had vanished. We sent our bill to the Bishop of Münster and he made good on it.

Now I was working in Bremen, I came and went as a decent journeyman in and out of the Herberge in der Tiefer, whose innkeeper was Father Roesemann. The things I saw the journeymen do there really made me sick. Playing cards, drinking beer, that was all. I thought: this has to change. I got together with several friends and reminded them that we went out in the wide world to learn something. At the next official meeting I stood up and spoke, after the regular business was taken care of and addressed the meeting attendees in the following way: ``We all went out into the world to learn something, to educate ourselves in practice and in theory. So, that when we return home, we would become good masters and decent citizens. A carpenter has to be able to draw, write, count and keep books. I allow myself herewith to propose a continuing education club, and let's do it here in this inn, here we are at home, here no one can forbid us anything.'' Said, done! All agreed. An evening for the first meeting was designated. The club, which was given the harmless name of ``The Evening Entertainment'', was founded, a teacher was hired, statutes were drawn up and officials elected; myself of course as President. The lesson plan was as follows:

\begin{itemize}
\tightlist
\item
  \textbf{Sunday}: from 7 AM until 1 PM, drawing
\item
  \textbf{Monday}:evening 8 PM until 10 PM, bookkeeping;
\item
  \textbf{Tuesday} evening, 8 PM until 10 PM, mathematics;
\item
  \textbf{Wednesday} off;
\item
  \textbf{Thursday} evening, from 8 PM until 10 PM, German
\item
  and \textbf{Friday} evening, from 8 PM until 10 PM, Geometry;
\item
  \textbf{Habicht:} off.
\end{itemize}

And this all went along really well. All journeymen showed up eager to learn. Every four weeks, there was an evening for debate. The best teachers came and gave us lessons.

In the spring of 1852, I again got sand in my shoes and decided to travel together with three others, a tailor, a shoemaker, and a carpenter, to Amsterdam. We celebrated our leaving; everyone promised by handshake to keep our project going and to keep the continuing education club running. A large number of people accompanied us to the outskirts of Bremen; for hours walking on a country road, we sang another song and then said goodbye. We were again wanderers. By way of a part of Oldenburg and Hannover, we arrived on Dutch territory and got aboard a ship in Harderwijk near the Zuiderzee. After a very stormy and dangerous trip, we arrived one evening in Amsterdam. In front of us were three brightly lit streets; I said: We'll take the middle one, and after we had walked down the middle street quite a piece, someone called to us from a wine cellar: Germans! Come in here! I am a young man from Bremen, a good German; here you'll be treated well. We looked at each other and in we went.

We had nothing to be sorry for there. We were treated very well there. The next day we went to look for work and also found it. Well, to describe Amsterdam would fill more than a book; here are just a few things. The town is built right in the middle of water; all houses are on poles; the city hall is built on 60,000 poles. All streets have canals, through which the ships glide with full sails. There are only two solid places in town, the Butter Market Place and the Place of the Mayor's house. Six hundred bridges took care of traffic flow. And the population, one can hardly say, if there is a pack of people who are more possessed by hate for the Germans. Whenever we entered a public restaurant, sat down to get something to eat, everyone moved far away from us. - Muff! Muff! they called us; that is the rude name for the Germans. When we had a glass of wine which, by the way, was very small, they all yelled: ``Can these guys drink'' and they too themselves drank schnapps by the quart. One half of the population lived in huge palaces covered with silk, jewels, and money, and the largest part of the population was poor. So poor, which we Germans could not even begin to understand. There were places to sleep which consisted of a tightly stretched rope over which the poor were hanging, sleeping. - With a wheelbarrow equipped with some type of rack, on top of that an iron plate and a few pots and pans, they went from place to place and kept up this moving pitiful restaurant going; below the heads of little children were peeking out, and at night they would stop at a place protected from rain and storm. Many people there live this way. Drinking schnapps, giving in to their worst passions, complaining about anything German, you see, this is what we found in Amsterdam. No wonder that it filled us with disgust and astonishment and so we left again after experiencing quite a few unpleasant things. Still in Utrecht, not far from the German border, they ran after us, Muff! Muff! they called us. As we arrived at the inn to spend the night, everyone stared at us and the women screamed: Oh, the poor German boys, they want to get around in the world, and at home they have nothing to eat. Our boys have it made compared to them, they can stay at home, they do not have to starve. We had to endure all this in silence and we were so happy the next morning when we were standing on German soil and saw theold father Rhein (river).

Germany! Germany above everything! we screamed out of a full heart, off we went. There we set out above the Rhein and went from there to Bonn ``the university city,'' until we could greet father Rhein again at Mainz. In Mainz, a man took us to Gernsheim am Rhein, and thus we went with him on the steamer, rode there overnight, shoemaker, tailor, and carpenter, in order to get work there. After we had been there some time, we got sand in our shoes again. The tailor stayed there, the other carpenter, my fellow countryman, went to Frankfurt a.M. and took off with the shoemaker to travel to Switzerland. First, we went to the Rheinpfalz, to the beautiful country where the grapevines are blooming. At Karlsruhe, we took the train. Arriving near Freiburg in the Breisgau, police boarded the train, took our travel logs with the remarks that we could get them back again in Freiburg at the police station. When we got there, the word was ``Turn back! You are not permitted to travel to Switzerland.''

The shoemaker took off to Wuerttemberg and I went to work in Freiburg. The poet sings about the Breisgau rightfully: The most beautiful region in the whole land - You are a nordic paradise, etc. It is truly a blessed area. The aftermath of the revolution was still noticeable. At eight o'clock in the evening, everyone had to be at home. Every half hour a patrol went through the streets, one time military, one time citizens' army, one time firemen. Whomever they found in the streets after eight o'clock received an overnight stay, a great big meal with plenty of beer and wine, and everything very cheap. Then suddenly, when I felt quite comfortable, a letter arrived from Bremen, it was from the Club, the Evening Entertainment. They begged me to come back to Bremen immediately. I wrote back to them. If you send me traveling money, I'll come! And see there, by return of mail, I received ten Thalers.

Well, then. Good bye you beautiful Badener land. Off! I went via Karlsruhe, Mainz, Frankfurt a.M., Kassel etc., again to Bremen. There was great joy when the President of the Club returned. I went back to work there, fulfilled my duties towards the Club faithfully, everything went smoothly. Our Club already had a reputation, made itself a name, famous men visited us, were delighted and surprised to find such an active, progressive Club life. Since the reaction everywhere was in full bloom and that was our ruin. - It was the holy night before Easter 1853, when I arrived at the inn during the afternoon. As I arrived there, they hurried towards me. ``Anton, they are looking for you already the whole day.'' I said: ``Me? Who is looking for me?'' ``The police! Hurry up and get out of Bremen before the police catch you and lock you up.'' I said: I'm not leaving Bremen, I have not committed a crime, and only a bad shepherd leaves his flock.'' ``See, there. They are just coming back again, four policemen. They do not know you personally, so just walk right in between them and out of town.'' Said, done. I walked without any worries right between the policemen, promised myself though one more time not to leave the city. I looked in my room for my suitcase. The innkeeper told me that the police had already picked it up in the morning, and thus I went straight to the police. I said to the officer on guard: I am a carpenter journeyman August Anton from Zerbst. Come in, he said, we have been waiting on him for a long time; come in here, and then I was inside four walls until the magistrate was called I am a carpenter journeyman August Anton from Zerbst.

``Come in,'' he said, ``we have been waiting on him for a long time; come in here,'' and then I was inside four walls until the magistrate was called. I don't want to bore you with the endless interrogations just this much, it was eleven o'clock in the evening when the judge got up and declared solemnly:

``The carpenter journeyman, August Anton from Zerbst, is herewith arrested and will be led away to the Oberthor. Attention. Forward, march!''

I was told. When we arrived there at the office all my pockets were searched and all my money taken away. Then they said,

``No.16 - attention! March! Stop! Here is No.~16, stay to the right, there is your bed.''

Bang, the door shut locked. I stood there as if I was touched by thunder and said:

``Good Evening!''

``Good Evening!'' it echoed from the corner on the left. I said,

``With whom do I have the honor of being together?''

``I am an arsonist! And you?''

I answered:

``I am a political criminal!''

``Well, I guess then we'll get along well together. Go to bed now, tomorrow we'll look into each other's face,'' he said.

Morning came, we looked at each other, and got along very well during the four weeks we spent together. He was a ship's captain, born on the sea and rode across all the oceans for 36 years. Now the trouble started. In the middle of the night they came and brought me in to be interrogated. They tried everything possible to get me to point the finger at other people who tried to use us for their political undertakings. We were supposed to have corresponded with the Death Gang from Mecklenburg and were supposed to have done all kinds of shameful deeds to rekindle the revolution. The masters all cried - oh dear me -, because all their journeymen were locked up, 66 journeymen, common members and eight Board members. Well, I'll make this short. After approximately four weeks, it happened this way one morning:

``Up, in front of the Senate to the General Meeting.''

So, we were transported there. The entire hall was full of people. The torture began anew. After lengthy cross-examinations the president rose and said:

``The verdict by the Senate against the carpenter journeyman August Anton from Zerbst is six months in jail! If you still have to say something, you herewith have the floor.''

I stood up and said: ``Yes, Mister President, I have something to say.''

``Speak.''

I said: ``Mister President, high Senate of the City of Bremen! - When I still attended school, I was told, boy, learn something so you will find success in the world. This I have sincerely done! I breezed through all classes and was confirmed with honor in the first class. When I started my apprenticeship thereafter, I was told again, boy watch out, listen to your master. Learn your trade well so you'll be successful in the world. I have done this sincerely, too. After my apprenticeship was over I have done the journeyman part, I passed the exam, and the same official who gave me the exam and declared me a journeyman, gave me his hand and said: You are now a journeyman, now you will go to other towns, other lands and seek to perfect yourself so when you return, you can become a capable master and a good citizen. Well, with this intention I came to the free city of Bremen. What did I find here, the journeymen sat at the Inn during their free period and played cards, drank beer and smoked. This cut deep into my soul. I said: Journeymen! You went out into foreign lands to learn something. If you want to become masters some day, you should be able to do math and keep books in order to be able to run a business successfully. I make a suggestion to you, we will found a Club and use our idle time to gather knowledge.''

They agreed. We have founded such a Club and truly done our duty. I thought if\ldots I will leave Bremen and my activities would become known to the high Senate, I would be receiving an award, instead I find myself subjected to interrogations and as punishment even sentenced to a dishonorable jail term. That's all I have to say. Total silence followed. The men put their heads together, the President came forward and said:

\begin{quote}
The sentence for the carpenter journeyman August Anton is herewith lifted, however, he'll be taken across the border by a policeman, exiled from Bremen for 99 years and ordered to return to his homeland.
\end{quote}

Two policemen took me and the VP of the Club, Herrmann Franke, from the courtyard of the Dom to the middle between them and off we went. As we crossed the market, it was filled with people. They formed a path and as we marched through, they all yelled an enthusiastic hurray!

\begin{quote}
Live well, live well! they screamed. You'll come back again to Bremen.
\end{quote}

We were led across the bridge and then a piece further, then our escorts stopped and said:

\begin{quote}
Now, we think you are brave guys, you'll find the way across the border by yourself and won't cause us any problems and come back to Bremen.
\end{quote}

We said, we would not even think of it.

\begin{quote}
Well then, give us your hand and have a good trip. Goodbye! Promise.
\end{quote}

After we had walked a short distance, there was a sign which let you know that beer was sold there. We went in and exchanged our feelings over a glass of beer.

\begin{quote}
Hermann, I said, I am going back to Bremen again, I have to see Marie again and since we were of opposite opinions about going back to Bremen, we said goodbye to each other.
\end{quote}

I went into a workshop where they made chairs while I was in the newer part of town. Everybody there was astonished, they formed a circle around me and asked me what they could do for me. I said:

\begin{quote}
Give me some money and different clothes, because I want to go into Bremen one more time, so that the police won't recognize me.
\end{quote}

In no time I had everything, a handshake and forwards I went across the bridge to the Inn again. Father Roesemann was astonished,

\begin{quote}
Hurry, he said, come into my private room.
\end{quote}

I sat there until it got dark, then I went to see her. We said goodbye, looked in each other's eyes once more, and then it was time to say goodbye here too.

The next morning, early at five o'clock, I left in the company of a friend, Heinrich Deyen from Melle, Bremen and we marched 17 more hours to Nienberg on the Weser (river). The next morning we said goodbye to each other, he went to Melle and I went via Hannover, Braunschweig, and Magdeburg to Zerbst. My parents were amazed that I showed up all of a sudden.

The next morning I showed up at the police station. Everyone shouted:

\begin{quote}
Welcome Mister President!
\end{quote}

They took away my passport for exile and informed me that I will remain under police supervision for one year and that I could not leave town.

A short time later I became very ill for several months and I have to thank my deceased mother for her care for me, otherwise I could have died. Later, I worked again for a local master, met stage stars and acted as a comedian because I was bored. When I had enough of that, after I had helped to bury my dear father in 1854, I went in the year 1855 to Berlin, worked there another three years, and had to admit to myself that I have been a real jackass for not having come to Berlin earlier. There you always worked in the same workplace and, therefore, you became very skilled in the trade and earned a lot of money.

My thoughts were now directed to becoming a master in Zerbst and so I went to draw each time I had saved 100 Thalers, and when the money was gone, I began to work.Again. 1858 I went home and passed my master exam and started my own business. Since I was the master now, I had workers who wanted to have food and drink and my mother was already too old to keep house, I was forced to look around for a Mrs.~master. Then I finally found her, Bertha Trebitz was her name; to see her and to love her was one and the same. We lived well and comfortably together and shared equally the joys and sorrows of this life.

I would like to tell a little bit about the Club life in those days. We had a sports club to which all the different social classes belonged, except the military and the ministers. 800 members out of a town with a population of 10,000---that was quite a big part. Gymnastics were done on a regular basis, the best teachers were here and always after the hour of physical exercise was over, an hour of drinking followed where jokes and humor were never absent. To play cards was forbidden. The festivities were not scarce either. We belonged to the Exercise League of the Saalgau, which stretched from Leipzig to Magdeburg. And in the summer months there was a founder's day festival held in one or the other town which we always had to attend. When we marched into one of these festival towns, all the children stood there with ``room for rent'' signs and yelled: ``My mother wants a sportsman too.'' Well, we then went with them wherever the choice took us and spent beautiful times.

In 1863 we had the founder's day festival, Sunday July 2nd, to which all league member clubs came. I had to hold the welcome speech. High on top of a stage at the open marketplace in front of 20,000 listeners. The speech went like this:

\begin{quote}
``German sports brothers! Through my mouth all the sportsmen from Zerbst are welcoming you. This old, hospitable, festively decorated city greets you. Welcome, you brothers from all the united districts in Anhalt! Welcome you dear neighbors! All! All! You who came to strive with us towards the same goals. And this strive, carried out by a fresh, free, religious and happy mind, also shall fill our hearts and kindle our strive towards the highest virtues as Father Jahn and Father Arendt taught us: The love of the fatherland! The love of freedom!''
\end{quote}

---Etc. After I spoke, a cheer high to the heavens sounded and all, professors of eloquence, shook my hands and said: ``Yes, who else can do it like this.'' This was a sports club in the truest sense of the word. We had other clubs too, workers' club, consumers' club, in which I was the stock keeper. As such I was sent to Magdeburg in 1868 to a General Meeting of the league and there I met Leberecht Uhlig, the enthusiastic fighter for religious freedom. I belonged to the board of a loan club, it was organized according to the principles of Schultze Delitzsch, and each Thursday evening at the meeting I spent a nice evening, a keg of beer was always there and the dignitaries like Mister Mayor and such were all members there. Until the big bang came and the cashier, Gustav Partheil was his name, took off with 40,000 Thalers. The pain and sorrow were great.

My dear mother died in 1868 and shortly after that my wife said to me: ``August, I don't want you to touch another plane, there is no money in it.'' I looked at her surprised and asked: ``Well, what do you want to do with your children now? Do you all want to starve?'' ``That's my business,'' she said. She borrowed money, bought a house in the best business district of the city, started a merchant's store, here called a ``grocery'' and then, when she did not like that anymore she opened a restaurant, andSince there was the need for a tourism office in town, I set one up. So, we had everything we wanted, earned tons of money. I was able to sell schnapps, beer, wine, tobacco and cigars in or out of my home, so I gave in.

We sold everything, took our six children and registered ourselves in the beginning of November 1871 aboard a ship to New York, and from there to Muskegon, Michigan. Here, everything went well too, until in 1873 an old acquaintance, the stork, flew on top of our house and my dear wife bled to death at the birth of a boy. So I stood there, nearly out of my mind, with my seven children. What good did it do, she was not going to become alive again, so I had to get myself together and take care of the children. I bought three acres of land near the town, built on it a beautiful house, started a garden planted with fruit trees and grape vines, and planted vegetables. The children were able to sell it all very easily and we were well off again. The social circumstances there were the utmost best. There, we had a blossoming German sports club, the sportsmen from the neighboring towns visited us, and we went there too; it was a truly German life. We also had a theatre club, where we presented German comedy. Everything ran beautifully and with joy. And again and again I thought about change. The beautiful, sunny South had gotten to me and so, in January 1877, leaving the other children home, I left with my son August to go first to the beautiful, intimate Cullman, Alabama.

Father Cullman met us at the train station and took care of a place for us to spend the night. The next day I was given a house to use, free of charge; we both worked at the ``Dreher's Furniture Factory'' and when the work ran out, we worked soon here and there. It happened then one day, as I worked at the Richter's Hotel, that I got a hold of a newspaper advertisement from Tuscumbia, an ad which was very tempting for Germans. I went there and a certain Roth took me out to the Spring Valley, about seven miles away from Tuscumbia, to a certain Captain Raether. He had for rent a 600-acre farm on which there was also a flour mill. This was a very tempting opportunity for the right man. The house was like a castle, a huge park around it, ground the park hundreds of acres of the most fertile land, an enormous herd of cattle, four horses, twelve cows, 600 sheep, hundreds of pigs and chickens, uncountable ducks and geese. Also a 36-acre fruit garden. All this for the price of 500 dollars a year. Thirty acres were divided into lots, these alone would bring in the rent. I said: Captain, I am a craftsman, I cannot handle all this, but I know a man in Cullman, he has money and is a German farmer, I will get him and then we will rent this farm. Said, done. I returned to Cullman that same evening. My son August waited for me at the train station and so we went immediately the same night.seven miles out to the farm where Hans Mix lived, the same man I told Captain Raether about. He agree.

Tuscumbia then. Packed his things the next morning and off we went to contract in. When we arrived on the holy night before Easter 1877, we signed the contract where Captain Raether was present with the realtor and off we went to Spring Valley. We began to work the farm. I let two of my children, Fritz and Ann, come from Michigan and everything was running smoothly at the beginning. The ``Flour Mill'' brought in groceries for us, which we needed, and milk, eggs, and meat we already had enough on the farm. But soon I realized that Mix was not the right man to run a big farm like this. Instead of setting himself on his horse like a German manager and to drive the niggers who worked for us, he himself worked like a nigger, and she too, was a dumb, tight, crazy female. I stuck in there till August, then I had enough of Mix's way of running the farm. I had the realtor Roth come out and asked him to release me from the contract. Captain Raether wanted to pay me for the work I had already done, but I said, I don't want anything, nothing at all, just want to be free and leave here. Then I was released from my responsibilities and went to Tuscumbia.

Here I began to work at a certain Challen, who owned a furniture store and also a workshop with a lot of machines. Next door was a grain mill and a cotton gin, which my son August took over; they belonged to Challen and Captain Raether \& Co.~We received a house to live in, with a big garden, all rent free. There was not much money, but we did not need that much and we were content. I had my other children, Anna and Emil come from Michigan and now we were all together again. We met good people there, i.e.~the family Bressler, where I worked quite often. Everything continued like this till the spring of 1879. A Jewish merchant lived there, named Levy; he had been in Birmingham and came to me and said, go to Birmingham, I was there and have seen it; there is a place for you, for a capable German craftsman. Since I had traveled around the world quite a bit, at first I did not want to pay attention, but he did not leave me alone, he came early in the morning, come let's go have something to drink so you'll get courage; I can hardly stand by and watch you rot in such a miserable place. Now, edging on helps and so I left again with my son August and took off first for Birmingham.

After arriving there we first worked at the Rolling Mill Cottages which were under contract with Frank P. O'Brien, who in those days had a hardware store. We were boarders at John Wiesendanger's, there I met August Grueninger. He said to me, don't be so stupid and don't work there anymore, I'll get you a better job, where you'll be your own ``boss'' and earn more money; be at the corner of First Avenue and 20th Street in the morning early at eight o'clock. I was there at the time set and he went with me to the office of the First National Bank and introduced me there to Colonel Linn, the same gave me work and first I had to build a house on Fifth Avenue, which he was building for his daughter Mrs.~Watson. I worked for him year and day, received my money every Monday morning like clockwork, and he would say that if he gave me the money on Saturday evening I would spend it all on drinking and to prevent this, he pays on Monday morning. Now, everyone has his own opinion. He meant well and had I listened to him, maybe I would be a wealthy man today. I wanted to build me a shop next to the bank, First Avenue, there where\ldots i.e., water and that we had any, but I did not want it because there was so much.

My luck would be good for a cabinet maker's shop. Had I done it, set myself a shop and house. I took care of work and everything else. So the lumber yard is today, after he had out the street south, there were the Reynolds. The lot belonged to the Elyton Land Company, store equipment 140 dollars.

Dollars per monthly payment and eight percent interest per year, the property was sold for 5000 dollars---I preferred the latter. A few years later, the property was sold for 10,000 dollars and I had to pay 10 dollars rent. In 1887, somebody bought the property for 27,000 dollars and I had to pay 20 dollars a month for rent.

On my request to be allowed to move possessions to another place, he replied: No! In 1889 I had to build an oak partition for the client company. When it was finished and found to be okay, the company said to me: Call all your sons together. I'll give you a house, I'll give you a good supply of lumber stock, and I'll give you the necessary capital. I want you to run a business of a larger size.

We got the house, two months free, then 50 dollars rent per month, and later 80 dollars per month. We received the machines; they were good. We received one car load of lumber, which was unusable except for putting it in the boiler. Money, never a cent, only ever so often when the need was the greatest they had to honor an overdue bill.

We had a lot of work. My son August was a terrific foreman. We had 20 to 30 people working for us. When Saturday came and the people did not get their money, there was trouble. The money was owed, but one cannot get it just like that, and so my son August got mad and said: I would rather work for wages as foreman. He went to Memphis, and this situation was finished. Then the Southern Wood Working Company was built on the ruins. Each co-worker was supposed to put up 200 dollars of their own capital. That company went bankrupt too, later. So I became partner with an Italian named Snetti, built our own shop on 20th Street No.~407 and then we felt independent again and everything went well. My partner left me later, started his own business, and soon after died. Then my sons Emil and Fritz said to me: Father, give us the shop, you are old already and should not have to work so hard, you stay with us. And it remained like that till today.

Now, to speak about the social life here in Birmingham. In the year 1881, a number of German men got together and talked about us Germans here in Birmingham wanting to hold a festival to prove that the Germans can get something going too. And so we held a festival at the South Side, a festival like they had never seen before in Birmingham. We built a theatre at the place where the Reynolds Lumber Yard is today. We built a dance hall and a playground. In the evening, there were big fireworks. Finkenheimer, deceased, was the manager. Friend Ruppert delivered the wood to build everything with. He recently wanted to build himself a house and shop nearby. On Monday, October 10, was the big day. All of Birmingham was represented, and everybody had great fun. A German preacher passing through Birmingham held the speech. When the festival was over, having done well financially, it was said: Germans stick together, and so we founded a Protestant church community. Pastor Thuemen, whom I had met in Tuscumbia, was the first.We held our meetings at Professor Paling's and his unforgettable wife's house.

The next thing was to start a sports club. An already 73-year-old tailor named Landgraf was a sports teacher. Oh, that was a life. Someone had provided us with a sports hall near where the brewery is today. When the cashier later took off with 18 dollars, the thing got a push. For a while, everyone was sleeping until we got together later one evening at August Dunker's, who ran a store on 14th Street in those days, and we started again to found a sports club. Friend Weidemann became the coach. The first sports hall was located on the South Side. Then in the place of the German Association ``Sublet Hall'', and then we moved to First Avenue. There we did sports and held festivities until the end came there too. Unpleasant events appeared to make it plausible that the club suspend its activities. Such a decision was made by the Board and the utensils and all equipment were stored in the Brewery Schillinger until after two years a reorganization finally occurred. So now, there was just a blossoming German Club here, the German Association, a club full of life and activity.

Several members quit this club in '86 and founded a second German Club, they said Concordia should be its name. Then the Sports Club began to come alive again under President Emil Lesser. Germans relaxed there; sang, happiness and German traditions were practiced. And it is this way to this day. May God, being good-willed towards the Germans, keep these two German Clubs for a long time here in Birmingham, so that both can blossom and grow and give witness to that wherever the Germans might live, German song and German traditions will be practiced by them. I have the honor to be a member in both clubs for many years, even as an Honorable Member in one of them.

Well, even though I have spent many long years in beautiful Birmingham, I have not been able to acquire money and property, but the love of my children and the respect of many of my dear friends and countrymen remained, friends from whom I had often gotten quite real proof of their love through actual help and devotion through bad times and illness. My most sincere wish is that it may remain this way until the end of my life. Well then, I probably could write about a lot more, but everything has to come to an end one day and so I have only one wish that these, my written-down memories of my life, would please my German friends and countrymen. Thanks are also due to the German Religious Societies, they stand for discipline, education, and humanity, the most important factors in the life of people and I only wish that my health will allow me to spend time in their midst quite often, though I can hardly ever go there, I am always among them in spirit, being the good Christian I am.

August Anton

\textbf{Epilogue}

These reflections were initiated on the advice of many friends of mine.

The publication of the work was thought to find with it a way to create a source of income for the writer.An old man, who was credited with the creation of the Germany here, without having to appeal directly to the sense of welfare of the Germans. The booklet which was to be sold for the humble price of 50 cents, is worth the price in full and offers to most people a most interesting and welcome reading material which, because it is written with suspense and in colorful pictures out of people's lives, will never bore the reader and should, therefore, be found in every German household in Birmingham. In the interest of the now 70-year-old man I am, therefore, expressing the wish of many of his friends that our fellow citizens may help him, please, by purchasing one booklet in order to brighten the last days of the life of the writer.

Emil Lesser
